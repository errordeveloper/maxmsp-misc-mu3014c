%!TEX TS-program = xelatex
%!TEX encoding = UTF-8 Unicode

\documentclass[12pt]{report}
\usepackage{geometry}
\geometry{a4paper}
\usepackage{graphicx}
\usepackage{amssymb}

%\usepackage[numbers]{natbib}

\usepackage{chngcntr}

\usepackage{pst-sigsys}


%\def\NOHREF{}

\ifdefined\NOHREF
\usepackage{url}
\newcommand{\URL}[1]{\[ \texttt{\emph{#1}} \]}
\newcommand{\href}[2]{#2 (\texttt{\emph{\url{#1}}})} %% not needed probably
\newcommand{\Href}[2]{{#2}} %% fake version for printing
\else
\usepackage{hyperref}
\newcommand{\Href}[2]{\href{#1}{#2}} %% use this for proper href
\newcommand{\URL}[1]{\[ \Href{#1}{\texttt{\emph{#1}}} \]}
\fi

%% use \Href  and for printing we use \cite{something}
%% so we end up with link to the URL and bibliography
%% reference number as well. If \Href is fake there
%% will be no URL in the text.
\newcommand{\AltRef}[3]{\Href{#2}{#1} \cite{#3}}

\usepackage{fontspec,xltxtra,xunicode}
\defaultfontfeatures{Mapping=tex-text}
\setromanfont[Mapping=tex-text]{Hoefler Text}
\setsansfont[Scale=MatchLowercase,Mapping=tex-text]{Gill Sans}
\setmonofont[Scale=MatchLowercase]{Andale Mono}

\title{Digital Reverberator Design in Max/MSP\\using Schroder Algorithm}
\author{Ilya Dmitrichenko
\\\small DSP Programming Case Study (MU3014C)
\\\small London Metropolitan University}

\date{\today}

\begin{document}
\maketitle

\counterwithout{section}{chapter}

\section{Introduction}

  \subsection{Project Organization}

  As a matter of good practice, a source code revision control system had
  been used in the course of this project. The file format that is used by
  \emph{Max}, is of plain text JSON (JavaScript Object Notation) type, which
  is perfectly suited for use with any revision management system as opposed
  to audio files, which are usually of relatively large size and not very
  much suitable for use with regular version management systems. 

  The revision control tool of choice for this project was \emph{Git}
  \footnote{More information can be found on \emph{Git} homepage:
  \URL{http://git-scm.com/about}}, moreover, in addition to a great set of
  benefits of using \emph{Git} itself, the \emph{GitHub} service enables an
  excellent web integration with simple to use user interface. The entire
  work history of this project can be examined on-line:
  \URL{https://github.com/errordeveloper/maxmsp-misc-mu3014c/commits}
  There is no need for the reader to be familiar with how to use \emph{Git},
  since the archive of the current version of the code can be downloaded
  from \emph{GitHub}:
  \URL{https://github.com/errordeveloper/maxmsp-misc-mu3014c/downloads}

  % Further in this report any files will be referenced  ...

  In the project's top-level exists `\emph{\texttt{patchers}}` directory, that
  is where most of the files of interest are located, unless explicitly specified,
  any of file names mentioned in this report can be found there.


  \subsection{}

\section{Implementation}
  \subsection{Sources}

  \cite{gardner1998algorithms}

%% Test bibliography:
\cite{farnell2010designing, moorer1979about, puckette2007theory,
gardner1992virtual, gardner1998algorithms, puckette1982reverb,
gardner1992reverb}

%%%%%%%%%%%%%%%%%%%%%%%%%%%%%%%%%%%%%%%%%%%%%%%%%%%%%%%%%%%%%%%%%%%%%%%%

\bibliographystyle{acm}
%\bibliographystyle{agsm}
%\bibliographystyle{plainnat}

\bibliography{MU3014C_coursework}

\end{document}  
