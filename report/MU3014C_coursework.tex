%!TEX TS-program = xelatex
%!TEX encoding = UTF-8 Unicode

\documentclass[12pt]{report}
\usepackage{geometry}
\geometry{a4paper}
\usepackage{graphicx}
\usepackage{amssymb}

%\usepackage[numbers]{natbib}

\usepackage{chngcntr}

\usepackage{pst-sigsys}


%\def\NOHREF{}

\ifdefined\NOHREF
\usepackage{url}
\newcommand{\URL}[1]{\[ \texttt{\emph{#1}} \]}
\newcommand{\href}[2]{#2 (\texttt{\emph{\url{#1}}})} %% not needed probably
\newcommand{\Href}[2]{{#2}} %% fake version for printing
\else
\usepackage{hyperref}
\newcommand{\Href}[2]{\href{#1}{#2}} %% use this for proper href
\newcommand{\URL}[1]{\[ \Href{#1}{\texttt{\emph{#1}}} \]}
\fi

%% use \Href  and for printing we use \cite{something}
%% so we end up with link to the URL and bibliography
%% reference number as well. If \Href is fake there
%% will be no URL in the text.
\newcommand{\AltRef}[3]{\Href{#2}{#1} \cite{#3}}

\usepackage{fontspec,xltxtra,xunicode}
\defaultfontfeatures{Mapping=tex-text}
\setromanfont[Mapping=tex-text]{Hoefler Text}
\setsansfont[Scale=MatchLowercase,Mapping=tex-text]{Gill Sans}
\setmonofont[Scale=MatchLowercase]{Andale Mono}

\title{Digital Reverberator Design in Max/MSP\\using Schroeder Algorithm}
\author{Ilya Dmitrichenko
\\\small DSP Programming Case Study (MU3014C)
\\\small London Metropolitan University}

\date{\today}

\begin{document}
\maketitle

\counterwithout{section}{chapter}

\section{Introduction}

  \subsection{Project Organization}

  As a matter of good practice, a source code revision control system had
  been used in the course of this project. The file format that is used by
  \emph{Max}, is of plain text JSON (JavaScript Object Notation) type, which
  is perfectly suited for use with any revision management system as opposed
  to audio files, which are usually of relatively large size and not very
  much suitable for use with regular version management systems. 

  The revision control tool of choice for this project was \emph{Git}
  \footnote{More information can be found on \emph{Git} homepage:
  \URL{http://git-scm.com/about}}, moreover, in addition to a great set of
  benefits of using \emph{Git} itself, the \emph{GitHub} service enables an
  excellent web integration with a simple user interface. The entire work
  history of this project can be examined on-line:
  \URL{https://github.com/errordeveloper/maxmsp-misc-mu3014c/commits}
  There is no need for the reader to be familiar with how to use \emph{Git},
  since the archive of the current version of the code can be downloaded
  from \emph{GitHub}:
  \URL{https://github.com/errordeveloper/maxmsp-misc-mu3014c/downloads}

  % Further in this report any files will be referenced  ...

  In the project's top-level exists \emph{\texttt{`patchers`}} directory, that
  is where most of the files of interest are located, unless explicitly specified,
  any of file names mentioned in this report can be found there.


  \subsection{Approaches to Reverberator Design}

  Studying the papers which describe various reverberation network topologies,
  it becomes absolutely clear that the degree of experimentation involved in
  the design process is usually very high. According to Jon Dattorro's article
  in the Journal of Audio Engineering Society (1997) \cite{dattorro1997effect},
  where he included a letter from Barry Blesser, who refers to the conversation
  with Manfred Schroeder in the late 1970s where Schroeder have said: \emph{"We
  did the electronic reverberation for amusement because we thought it would be
  fun. Since it took the better part of a day to do 10 seconds of reverberation,
  we only ran one sample of music. The notion of delay time selections was random
  in that we just picked a bunch of numbers and there was no mathematical basis.
  We just wanted to prove it could be done."} Also John Stautner and Miller
  Puckette in the article "Designing Multi-Channel Reverberators" (Computer Music
  Journal, 1982) \cite{puckette1982reverb} point out that \emph{"\dots no attempt
  has been made to imitate the ambience of an existing room or concert hall,
  the methods described herein may lead to such applications when they are
  combined with a consideration of the perceptual importance of attributes of
  the soundfield in a real room."}. However, the later papers by William Gardner
  \cite{gardner1992virtual, gardner1998algorithms} suggest formulas to calculate
  coefficients.


  \subsection{Brief Overview of History of Reverb Topologies}

  Manfred Schroeder had been a pioneer in artificial reverberator design, he
  worked at Bell Laboratories since mid-1950s and published an number of papers
  in the following two decades \cite{dattorro1997effect}. The original
  Schroeder's topology consisted of 4 parallel comb filters generating early
  echoes and 2 all-pass filters all in series \cite{gardner1992virtual}.
  Moorer suggests that Schroeder's design have exhibited fluttering decay
  and poor echo density \cite{moorer1979about}. The fluttering (sometimes also
  called "metallic ringing") can be observed in response to impulsive sounds
  (for example the snare drum), according to Gardner \cite{gardner1992virtual},
  though it is still certainly good for short reverberation times and moderate
  reverberation levels \cite{gardner1998algorithms}. One of the major disadvantages
  is that it \emph{"\dots does not provide a frequency dependent reverberation time"}.
  Moorer \cite{moorer1979about} has modified the comb filter block by adding
  two extra comb filters and a one-pole low-pass filter in the feed-back loop
  of each of the comb filters. The low-pass in feed-back was intended to
  dump the higher frequencies similarly to how those are absorbed in air.
  Moorer's technique has helped to reduce the unnaturally sounding flutter,
  however it has not eliminated it entirely. There had been a great amount of
  research in this area, where the two biggest problems had been - eliminate
  unnatural colouration and achieve the highest echo density. It should be
  noted that most of the later topologies had been designed by modifying the
  feedback path. For example, Puckette and Stautner \cite{puckette1982reverb}
  have achieved much higher echo density in their multi-channel system by
  using a rotation matrix in the feedback loop. Miller Puckette also presents
  this as an example\footnote{It had been a trivial task to implement Puckette's
  reverb in Max and it can be found in the project repository, stored as \texttt{`%
  \href{https://github.com/errordeveloper/maxmsp-misc-mu3014c/blob/master/patchers/reverb.matrix-rotation.maxpat}{reverb.matrix-rotation.maxpat}`}} in his book "Theory and Technique of Electronic Music"%
  \cite{puckette2007theory}.
  Gardner \cite{gardner1998algorithms} also highlighted a number of other
  various feed-back loop improvements as well as the work of Jean-Marc Jot in
  early 1990s, who introduced time-varying feed-back correcting algorithm,
  which deserves a separate case study and the author of this report certainly
  desires to do so. However, the subject set herein is the earlies reverberator,
  one which Manfred Schroeder has invented in early 1960s.


\section{Implementation}


  \subsection{Building Blocks}

  The building blocks of a reverberation network are also called unit
  reverberators and most commonly are at least  ...

  The building blocks of reverberators are filters, however these are slightly
  different from filters which are used for ...
  OR:

  The building blocks of almost any signal processing system are generally
  referred to as filters, there exit adders and multipliers too, however
  those are somewhat more trivial. 


  Looking into various source of the subject ..., it at firts seems that the
  comb filter ...

  It is noticably difficult to implement a arbitary algorith and obtain clear
  plain text output which can be copied into a report document, hence a small
  C program was written to prove that the above topologies of comb filters
  differ very slightly. Code listing \ref{comb:test:init} demonstrates the
  initialisation part of the program, while the listings \ref{comb:test:A}
  and \ref{comb:test:B} are the functions which implement the algorithms of
  each of the topologies under consideration\footnote{The entire program can
  be view on \emph{GitHub}:
  \URL{https://github.com/errordeveloper/maxmsp-misc-mu3014c/blob/master/tests/comb\_filters.c}}.
  The output of the program is given in \ref{comb:test:output} and plotted in
  figures \ref{comb:plot:A} and \ref{comb:plot:B}.

  \begin{figure} \begin{pspicture}[showgrid=true](-5,-1)(5,5)

 \psset{style=RoundCorners,fillstyle=solid,fillcolor=green!14,gratioWh=1.25}


 %--- Define Blocks:

 \pssignal          (-4,2.5)    {I}     {$x[n]$}
 \pssignal          (4.5,2.5)   {O}     {$y[n]$}

 \dotnode           (-3,2.5)    {dot0}
 \pscircleop        (-2,2.5)    {sum0}

 \dotnode           (1,2.5)     {dot1}
 \pscircleop        (3,2.5)     {sum1}


 \psfblock          (-0.5,2.5)  {del}   {$Z^{-m}$}


 \pscircleop  [operation=times](-1,1){g0d}
 \pssignal          (-1,0)      {g0}    {$-g$}
 \ncline{>-}{g0}{g0d}

 \pscircleop  [operation=times](0,4){g1d}
 \pssignal          (0,5)    {g1}    {$g$}
 \ncline{>-}{g1}{g1d}

 \pscircleop[operation=times](2,2.5){g2d}
 \pssignal          (2,1.5)    {g2}    {$1-g^2$}
 \ncline{>-}{g2}{g2d}


 %--- Link All Blocks:

 \psset{style=Arrow,fillstyle=none}

 \ncline{->}{I}{sum0}
 \ncline{->}{sum0}{del}
 \ncline{->}{del}{g2d}
 \ncline{->}{g2d}{sum1}
 \ncline{->}{sum1}{O}

 \ncangle[angleA=-90,angleB=180]{->}{dot0}{g0d}
 \ncangle[angleB=-45]{->}{g0d}{sum1}
 \ncangle[angleA=90]{->}{dot1}{g1d}
 \ncangle[angleA=180,angleB=145]{->}{g1d}{sum0}


\end{pspicture} \caption{\emph {All-pass Filter}} \label{allpass:A}\end{figure}

\begin{figure} \begin{pspicture}[showgrid=false](-5,0)(5,6)

 \psset{style=RoundCorners,fillstyle=solid,fillcolor=green!14,gratioWh=1.25}


 %--- Define Blocks:

 \pssignal          (-4,3.5)    {I}     {$x[n]$}
 \pssignal          (4,3.5)     {O}     {$y[n]$}

 \dotnode           (-2.5,3.5)    {dot0}
 \pscircleop        (-1.5,3.5)    {sum0}

 \psfblock          (0,3.5)       {del}   {$Z^{-m}$}

 \pscircleop        (1.5,3.5)     {sum1}
 \dotnode           (2.5,3.5)     {dot1}



 \pscircleop  [operation=times](0.5,5){g0d}
 \pssignal          (0.5,6)    {g0}    {$g$}
 \ncline{>-}{g0}{g0d}

 \pscircleop  [operation=times](-0.5,2){g1d}
 \pssignal          (-0.5,1)    {g1}    {$-g$}
 \ncline{>-}{g1}{g1d}


 %--- Link All Blocks:

 \psset{style=Arrow,fillstyle=none}

 \ncline{->}{I}{sum0}
 \ncline{->}{sum0}{del}
 \ncline{->}{del}{sum1}
 \ncline{->}{sum1}{O}

 \ncangle[angleA=-90]{->}{dot1}{g0d}
 \ncangle[angleA=-180,angleB=-90]{->}{g1d}{dot0}
 %\ncangle[angleA=-180,angleB=-90]{<-}{g1d}{dot0}
 \ncangle[angleA=-180,angleB=145]{->}{g0d}{sum0}
 \ncangle[angleB=-45]{->}{g1d}{sum1}


\end{pspicture} \caption{\emph {All-pass Filter}} \label{allpass:B}\end{figure}


%%%%%%%%%%%%%%%%%%%%%%%%%%%%%%%%%%%%%%%%%%%%%%%%%%%%%%%%%%%%%%%%%%%%%%%%

\bibliographystyle{acm}
%\bibliographystyle{agsm}
%\bibliographystyle{plainnat}

\bibliography{MU3014C_coursework}

\end{document}  
